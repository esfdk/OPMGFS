\chapter{Related Work}
\label{relatedwork}

There has been a lot of research on the subject of generation of maps for strategy games. The research has been of the more general type, but research on StarCraft map generation does exist, though it focuses on the original StarCraft.

A search-based map generator for the real-time strategy game \textit{Dune 2} is described by \citeauthor{Mahlmann2012Spicing}\cite{Mahlmann2012Spicing}. This generator utilises an evolutionary algorithm to search for maps and a cellular automata to convert found maps to actual game maps. The results were maps that both looked good and satisfied the specifications that had been set up, which proves that generating strategy maps is possible.

While being able to generate good maps is important, making sure they are diverse is an important quality. Maps that are simply variations of each other will quickly become stale. The research by \citeauthor{Preuss2014Searching}\cite{Preuss2014Searching} focuses on this particular aspect. They found that the distance measure had a crucial effect on diversity maintenance mechanisms and that novelty search did not reach the same diversity in generated results.

\citeauthor{Liapis2013sentient}\cite{Liapis2013sentient} introduces the Sentient Sketchbook, a tool that is intended to help designers create game levels. The Sketchbook alleviates designer effort by checking for playability and evaluating and visualizing significant gameplay properties. Constrained novelty search, an algorithm that searches for "newness",  is also introduced and used to provide alternatives to the designer's maps. This program is an interesting take on map generation because it attempts to cooperate with the designer, instead of taking care of everything by itself.

Something more in the direction of the goal of this thesis would be what \citeauthor{Togelius2010Multiobjective}\cite{Togelius2010Multiobjective} did. Knowing that a single fitness function will have a hard time properly describing something as complex as a StarCraft map, they used a multi-objective evolutionary algorithm in order to evolve maps instead. The use of the multi-objective evolutionary algorithm made the search computationally expensive and they found that it was difficult to find maps that were good on all parameters.

The research on multi-objective evolutionary algorithms with regards to strategy (and StarCraft) maps was continued, again by \citeauthor{Togelius2013Controllable}\cite{Togelius2013Controllable}. This time, they introduced a generic indirect evolvable representation for strategy maps and used a StarCraft specialized version to prove that it worked. More than a dozen evaluation functions were used and the generated maps were evaluated by skilled StarCraft players. Many of the evaluations partially conflicted, but some of the evaluations were demonstrated to correlate with interesting map qualities. This research was done with regards to offline map generation, and further work would be needed in order to achieve a speed that would be acceptable for online procedural content generation.