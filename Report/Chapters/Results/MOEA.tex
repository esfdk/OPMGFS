\section{Multi-objective Evolution}
\label{results_moea}

\begin{table}
	\begin{center}
	\renewcommand{\arraystretch}{1}
	\caption{Results of evolution with the NSGA-II algorithm.}
	\label{tab:results_moea_results}
		\begin{tabular}{| c ? c | c | c | p{1.9cm} | p{1.9cm} | p{1cm} |}
		\hline
		Settings & Max fitness & Average fitness & Min fitness & Convergence (average) & Convergence (mode) &  Time \\
		\hline
		A &  &  &  &  &  &  \\
		\hline
		B &  &  &  &  &  &  \\
		\hline
		C &  &  &  &  &  &  \\
		\hline
		D &  &  &  &  &  &  \\
		\hline
		E &  &  &  &  &  &  \\
		\hline
		F &  &  &  &  &  &  \\
		\hline
		G &  &  &  &  &  &  \\
		\hline
		H &  &  &  &  &  &  \\
		\hline
		I &  &  &  &  &  &  \\


		\hline
		 & & & & & & \\
		\hline
		\end{tabular}
	\end{center}
\end{table}

\begin{table}
	\begin{center}
	\renewcommand{\arraystretch}{1}
	\caption{Results of NSGA-II seeded with highest fitness novel individuals.}
	\label{tab:results_novelevolutionhighfitness}
		\begin{tabular}{| K ? S | S | S | L | L | K | K |}
		\hline
		Settings Combination & Max fitness & Average fitness & Min fitness & Convergence (average) & Convergence (mode) & Evolution Time & Novelty Time \\
		\hline
		E + IV &  &  &  &  &  &  &  \\
		\hline
		F + I &  &  &  &  &  &  &  \\
		\hline
		G + VIII &  &  &  &  &  &  &  \\
		\hline
		J + V &  &  &  &  &  &  &  \\
		\hline
		H + VI &  &  &  &  &  &  &  \\
		\hline
		I + VII &  &  &  &  &  &  &  \\
		\hline
		D + II &  &  &  &  &  &  &  \\


		\hline
		 &  &  &  &  &  &  &  \\
		\hline
		\end{tabular}
	\end{center}
\end{table}

\begin{table}
	\begin{center}
	\renewcommand{\arraystretch}{1}
	\caption{Results of NSGA-II seeded with highest novelty novel individuals.}
	\label{tab:results_novelmoeahighnovelty}
		\begin{tabular}{| K ? S | S | S | L | L | K | K |}
		\hline
		Settings Combination & Max fitness & Average fitness & Min fitness & Convergence (average) & Convergence (mode) & Evolution Time & Novelty Time \\
		\hline
		E + IV &  &  &  &  &  &  &  \\
		\hline
		F + I &  &  &  &  &  &  &  \\
		\hline
		G + VIII &  &  &  &  &  &  &  \\
		\hline
		J + V &  &  &  &  &  &  &  \\
		\hline
		H + VI &  &  &  &  &  &  &  \\
		\hline
		I + VII &  &  &  &  &  &  &  \\
		\hline
		D + II &  &  &  &  &  &  &  \\


		\hline
		\end{tabular}
	\end{center}
\end{table}