\chapter{Conclusion}
\label{conclusion}
In this thesis, a method for generating two-player StarCraft II map layouts is presented. The presented method uses a cellular automaton to generate base maps that are then searched using three different approaches. A definition of map quality is presented as being the balance between the two players and viable strategies available to them. The three approaches have been evaluated based on their runtime, their capability for generating novel maps, and the quality of the maps they generate.

The constrained novelty search approach produces more novel maps than the other algorithms, but generally produces maps of lower quality and suffers from high variance in the level of quality. The standard genetic algorithm produces maps of a much higher average quality than the maps produced by novelty search and does so with far less variance in almost all cases. The multi-objective evolutionary algorithm produces maps of a high quality that is comparable to the standard genetic algorithm, but the runtime of the algorithm is slower than its single-objective counterpart. Seeding the genetic and multi-objective algorithms with individuals found using constrained novelty search have significant impact on the runtime of the algorithms without showing significant improvements in quality of the maps produced. The best trade-off between runtime, novelty, and quality is found using the genetic algorithm seeded with random individuals. It is worth noting that both the quality and novelty of the base maps that are being searched has substantial influence on the quality and novelty of the generated maps.

The layouts generated using the presented method are not completely compatible with the StarCraft II map editor, which means that the maps produced cannot be directly input into the game. Fixing this problem should be the foremost priority in future work done on the project, as it is not currently possible to create a fully automated map generator for the game using the specific implementation presented in this thesis. Additionally, improving the fitness function used to evaluate the generated maps could potentially change the relation between the three tested approaches.