\chapter{Goals}
\label{goals}
Prior research on the topic of generating maps for StarCraft has been focused on generating the maps in an offline (for during game development) context. Most maps generated by these methods require the developer to implement and polish such that the maps can be used in the game. While this is certainly useful to developers as it reduces the amount of creative work that goes into creating maps, it still takes time for the developer to finish the map before it is accessible to players. If the developer shelves the game, the game will no longer receive new maps from an official source, which means play at higher levels of skill may grow stale. Powerful tools such as the StarCraft Editor provide tools for the community to develop their own maps, but because such maps are not from an official source, it is not unlikely that they will be rejected by the community. If a procedural content generation tool was accepted or developed by the game developer, it would be possible to provide the game with novel content without spending resources years after the game has been released.

Any tool developed for such a purpose would likely have to be implemented as an online (during runtime) map generation tool. While the idea of generating novel content is great, it is unlikely that most players would have the patience to sit for tens of minutes (or even hours, depending on runtime), even if precise map balance is an integral part of StarCraft gameplay. This means that any implementation would likely have to sacrifice some degree of quality (balance) and/or novelty in order to generate maps. The main goal of this thesis is to explore the trade-off between balance, novelty, and runtime that exists when generating StarCraft II maps. The secondary goal is to evaluate 

\section{Map Balance}
\label{goals_balance}
have to see what is written about starcraft balance elsewhere...

\section{One-to-One Content Generation}
\label{goals_representation}

\section{Optimal Trade-off}
\label{goals_tradeoffs}
In order to reach an optimal trade-off between balance, novelty, and runtime, it is important to consider what minimum requirements exist for each part of the trade-off. As previously mentioned, the generation of a new map should not take longer than players are willing wait for it be generated. Based on personal experience, we have chosen two different time limits for map generation: two and five minutes. These time limits constitute a very short and short break, respectively. These breaks happen naturally when playing StarCraft in ladder games, as it takes time to find an opponent between games. If players are patient enough to wait for an opponent to be found, it is not unlikely that they would be willing to wait the same amount of time for a map to be generated.



\textbf{Random thought: Coordinate representation may have been better but would work only for fixed size maps}

\textbf{Random thought 2: Gad vide om det er muligt at implementere algoritmerne direkte i StarCraft II Editor'en......}