\chapter{Goals}
\label{goals}
Prior research on the topic of generating maps for StarCraft has been focused on generating the maps in an offline (for during game development) context. Most maps generated by these methods require the developer to implement and polish such that the maps can be used in the game. While this is certainly useful to developers as it reduces the amount of creative work that goes into creating maps, it still takes time for the developer to finish the map before it is accessible to players. If the developer shelves the game, the game will no longer receive new maps from an official source, which means play at higher levels of skill may grow stale. Powerful tools such as the StarCraft II Editor provide tools for the community to develop their own maps, but because such maps are not from an official source, it is not unlikely that they will be rejected by the community. If a procedural content generation tool was accepted or developed by the game developer, it would be possible to provide the game with novel content without spending resources years after the game has been released.

Any tool developed for such a purpose would likely have to be implemented as an online (during runtime) map generation tool. While the idea of generating novel content is great, it is unlikely that most players would have the patience to sit for tens of minutes (or even hours, depending on runtime), even if precise map balance is an integral part of StarCraft gameplay. This means that any implementation would likely have to sacrifice some degree of quality (balance) and/or novelty in order to generate maps. The main goal of this thesis is to explore the trade-off between balance, novelty, and runtime that exists when generating StarCraft II maps. The secondary goal is to evaluate the different algorithms that we implement in order to explore the StarCraft map space (discussed further in section \ref{methodology_algorithmchoice}).

\section{Map Balance}
\label{goals_balance}
There are many possible ways to define balance in games. The most common balance evaluation in two-player competitive games is whether one player has an unfair advantage over the other. Assuming equal player skill, there are three elements have major impact on who comes out the winning in a game of StarCraft II: choice of race for both players, choice of map, and the strategies chosen by the players\footnote{It can be argued that choosing the correct race and strategy is part of player skill. They are, however, still worth noting on their own.}. Because choice of strategy is heavily dependent on the map that is being played, the map being played has the largest impact on the result, if there is no imbalanced between the different races available to players. Thus the map must offer equal chances of players winning, else it cannot be balanced.

Our definition of what constitutes a \textit{balanced StarCraft map} is based on the definition used by \citeauthor{uriarte2013psmage}\cite{uriarte2013psmage}:
\begin{quote}
	"... we will define a \textit{balanced map} as one that satisfies two conditions: a) if all players have the same skill level, they all have the same chances of winning the game, and b) in the case of StarCraft, no race has a significant advantage over any other race." - \citeauthor{uriarte2013psmage}\cite{uriarte2013psmage}
\end{quote}

In addition to giving players of equal skill an equal chance of winning and allowing no race to have a significant advantage over any other race, we expand the definition of balance by taking the strategies available to players into account. A map must have more than one viable strategy (no single strategy can dominate others) so that players who favour different strategies are not left at a huge disadvantage when playing the map. 

We aim to achieve this map balance by making symmetrical maps and by aiming to equalise the number of opportunities that each type of strategy has. Creating symmetrical maps for real-time strategy games guarantees that one starting position is not more favoured than the other. In order to equalise different strategies, we aim to make sure that the most direct ground path between starting bases is not too short (favours rushes) and not too long (favours playing the macro game). Additionally, we aim to place the nearest expansion base at a relatively close distance and other bases somewhat further away, such that strategies focused on the mid-game are viable without making it too easy to secure multiple bases. To prevent mobile units have too much of an advantage (or disadvantage), we aim to avoid having too few or too many choke points on the path between start bases and to not have too few or too many angles of approach for bases. A low number of choke points or very open areas around bases will benefit mobile units heavily, as they can run around and/or past less mobile units, and vice versa in the case of a high number of choke points and/or very closed bases.

\section{Direct Content Generation}
\label{goals_representation}
Part of the initial goal of this thesis was to produce actual StarCraft II map files in order to provide players with a "play \& plug" option after running the generator. This is not possible, however, since there exists no functionality to do so within the StarCraft II Editor, and the map file format has, to our knowledge, not yet been reverse engineered in a way that is applicable in this context. While it is beyond the scope of this thesis to produce actual StarCraft map files, all players have access to the editor in which they can manually create maps for play. We intend to make use of this feature by generating an image file that shows a tile-by-tile layout of the generated maps, so that players (and developers) can create the maps manually in the editor.

\section{Optimal Trade-off}
\label{goals_tradeoffs}
In order to reach an optimal trade-off between balance, novelty, and runtime, it is important to consider what minimum requirements exist for each part of the trade-off. Because the gameplay of StarCraft relies heavily on the balance and fairness between players and the in-game races, any generated must be balanced as described in section\ref{goals_balance}. If a map does not conform to the requirements for it to be balanced, it cannot be considered as a solution to the problem of StarCraft map generation. It is imperative that generated maps are balanced within predictable parameters, as the tool will not be useful in any type of competitive environment if even a small number of output maps are unbalanced.

As previously mentioned, the generation of a new map should not take longer than players are willing wait for it be generated. Based on personal experience, we have chosen two different time limits for map generation: two and five minutes. These time limits constitute a very short and short break, respectively. These breaks happen naturally when playing StarCraft in ladder games, as it takes time to find an opponent between games. If players are patient enough to wait for an opponent to be found, it is not unlikely that they would be willing to wait the same amount of time for a map to be generated.

The search algorithms must also provide maps that are novel when compared to other maps generated by the search algorithms. If generator can only produce a very limit number of novel maps, then it does not live up to its functionality as a creator of novel content and thus is unusable for that purpose. Therefore it is a requirement that the generator is capable of generating, at least, several hundred novel maps of medium to high quality. The StarCraft map space is immense, so it is likely that any algorithm capable of traversing a wide area of that search space will be able to produce the required level of novelty in maps.

While it is likely that there exists a number of combinations that could be considered optimal trade-off, the focus of this thesis is to tend towards trade-off where balance is a dominating factor.