\chapter{Results}
\label{results}
There are many different options in the search-based map generation approach presented in this thesis. In most cases, these options have an influence on the results generated by the search and/or the runtime of the search. The goal of this thesis is to compare different search methods, so the testing was focused on changing settings that affect the search algorithms instead of the settings that control aspects of the map.

This means that some settings have not been changed during testing. One of them is the fitness calculation settings. They determine how valuable different parts of the fitness function is and changing them between tests would make it really difficult to compare methods. The default values mean that the highest possible fitness, theoretically, is 67. It is very unlikely to be reached, as some parts of the fitness may conflict with each other. The lowest possible fitness score for a map with starting bases is -44.66, and -737 if there are no start bases in a map\footnote{No start bases means that a map literally is unplayable and thus should never be presented as the final result.}.

The cellular automata has a number of settings (e.g. generations, rule sets, map size), which have been kept the same for all tests. The map size is a static 128x128, which is common for smaller two-player maps in StarCraft II. In all tests, the cellular automata ruleset is a basic set of rules that focuses on generating roughly an even amount of each height levels in reasonably sized clusters. The map search options have also been kept the same for all the tests. These options control how many of the different items a map should be able to contain, what the chance of adding or removing an item is and in which area of the map items should be placeable. This means that one half of a map can contain a maximum of one start base, five expansion bases, 18 ramps, eight sets of destructible rocks, and one Xel'Naga tower. 

\section{Base Map Generation}
\label{results_basemapgeneration}

The base maps we create are, as mentioned in section \ref{methodology_ca}, made through the use of a cellular automata. Due to the random seeding we utilize, the quality of the base maps can vary greatly, though most of them are good enough to create StarCraft II maps from.

Shown in figure \ref{fig:results_basemapgeneration_goodbad} are two base maps generated through the cellular automata. The one to the left features four sizable areas of height-level two that all lead down to areas of height-level one. Assuming a start base is placed on either of these areas, the base will be pretty well-defended, as the map requires an attacker to head down to height-level zero to even get to the other side of the map. Any expansion bases are likely to be defendable as well, due to the same properties.

The map on the right is a good example of a bad base map. It has a few small areas of height-level two, all of which connect very poorly to areas of height-level one. This means that start-bases on height-level two will often result in an infeasible map, as it is difficult to create ramps that would result in a path between the two start bases. Start-bases on lower height-levels are less desirable, but not the largest problem this base map has. 

The largest problem is creating a path that links start-bases together almost no matter how they are placed. There are very few places where it is possible to place ramps in such a way that they would connect the top half of the map with the lower half, which greatly reduces the amount of places a start-base can be placed.

\insertTwoPictures{BaseMap_Good}{BaseMap_Bad}{A good (left) and a bad (right) base map.}{results_basemapgeneration_goodbad}

The two base maps shown are very different from each other, a trait which runs true for all the base maps we create. Figure \ref{fig:results_basemapgeneration_novelty} shows how different the generated base maps are. It shows the difference in height-level for every tile for every map. A tile is dark if a specific height-level was encountered on that tile on many maps. The less times the same height-level shows up in the same tile, the lighter the tile is.

\insertPicture{0.8}{BaseMap_Novelty}{A novelty map made based on ten generated base maps.}{results_basemapgeneration_novelty}

From the novelty map, it is easy to see that the base maps generated did not often share the same height-level in the same position.

%do they work?

%show 1 good

% show novelty of base map generation
\section{Novelty Search}
\label{results_noveltysearch}


\begin{table}
\begin{center}
\renewcommand{\arraystretch}{1}
\caption{Results of novelty search.}
\label{tab:results_noveltysearch_results}
\begin{tabular}{|p{1cm}| p{1.1cm}  | p{1.3 cm} | p{1.2cm} | p{1.4cm} | p{1.3cm} | p{1cm} ? p{1.2cm} | p{1cm}  | c | }
\hline
Gene-rations & Feasible Size & Infeasible Size & Add To Archive & Minimum Novelty & Mutation Chance & Neigh-bours & Acceptable \% & High-quality \% & Time\\
\hline
\end{tabular}
\end{center}
\end{table}

\section{Evolution}
\label{results_evolution}

% Average værdier oer okay, men base maps giver stor variance.

% Testede 10 forskellige base maps med forskellige indstillinger. Tallene er average af de bedste resultater for hvert base map.

For the standard genetic algorithm, we ran three different kinds of tests:

\begin{my_itemize}

	\item Genetic algorithm using a random initial population. The settings used for this test are shown in table \ref{tab:results_evolution_combinations}). The results are shown in table \ref{tab:results_evolution_results}. The chance of mutation was set to 30\% and was never changed.

	\item Genetic algorithm using an initial population made from the highest fitness maps created by constrained novelty search. These tests uses one of the set of settings shown in table \ref{tab:results_novelty_combinations} for the constrained novelty search in conjunction with one of the set of settings shown in table \ref{tab:results_evolution_combinations}. The settings and results are shown in table \ref{tab:results_novelevolutionhighfitness}.

	\item Genetic algorithm using an initial population made from the highest novelty maps created by constrained novelty search. These tests have been made in the same way as the tests seeding with the highest fitness maps. The combinations and results are shown in table \ref{tab:results_novelevolutionhighnovelty}.

\end{my_itemize}

Like the results in the constrained novelty search section, all numbers in the result tables are the average of running the algorithm with settings on ten different base maps. \textit{Convergence} shows the generation at which there was no change in neither maximum nor minimum fitness for three consecutive generations. It is not necessarily the same as the generation at which the maximum fitness recorded in the tables was reached.

We have attempted to keep the parents:children ratio close to 1:4 with 1 parent per 5 population size, though it is not exact for all of the settings setups. Furthermore, the odds of mutation happening was set to 30\% for all of the settings.

\begin{table}[!h]
	\begin{center}
	\renewcommand{\arraystretch}{1}
	\caption{The settings used in the standard genetic algorithm and MOEA. \textit{Parents} and \textit{Children} are only relevant for the genetic algorithm, as the MOEA does not have such parameters.}
	\label{tab:results_evolution_combinations}
		\begin{tabular}{| c | c | c | c | c |}
		\hline
		Settings Label & Generations & Population & Parents & Children \\
		\hline
		A 	& 5 	& 5 	& 1 	& 4 \\
		\hline
		B 	& 15 	& 15 	& 3 	& 12 \\
		\hline
		C 	& 45 	& 5 	& 1 	& 4 \\
		\hline
		D 	& 25 	& 10 	& 2 	& 8 \\
		\hline
		E 	& 10 	& 25 	& 6 	& 18 \\
		\hline
		F 	& 30 	& 20 	& 4 	& 16 \\
		\hline
		G 	& 5 	& 50 	& 12 	& 36 \\
		\hline
		H 	& 50 	& 5 	& 1 	& 4 \\
		\hline
		I 	& 50 	& 10 	& 2 	& 8 \\
		\hline
		J 	& 10 	& 5 	& 1 	& 4 \\
%		\hline
%		 &  &  &  &  \\
		\hline
		\end{tabular}
	\end{center}
\end{table}

\begin{table}[!h]
	\begin{center}
	\renewcommand{\arraystretch}{1}
	\caption{Results of evolution with a standard genetic algorithm.}
	\label{tab:results_evolution_results}
		\begin{tabular}{| c ? K | K | K | L | L | K |}
		\hline
		Settings & Max Fitness & Average Fitness & Max Fitness (std. dev.) & Convergence (average) & Convergence (std. dev.) &  Time (seconds) \\\hline
		A 	& 29.27 	& 21.50 	& 14.97 	& 3.30 	& 1.49 	& 5.94 	\\ \hline
		B  	& 40.66 	& 34.63 	& 4.41 	& 12.20 	& 3.33 	& 38.03 	\\ \hline
		C 	& 40.65 	& 36.78 	& 7.08 	& 25.10 	& 14.82 	& 35.22 	\\ \hline
		D 	& 42.65 	& 37.73 	& 3.08 	& 7.50 	& 5.36 	& 41.97 	\\ \hline
		E 	& 43.37 	& 30.81 	& 3.34 	& 7.00 	& 2.40 	& 47.56 	\\ \hline
		F 	& 42.91 	& 37.43 	& 3.75 	& 10.30 	& 5.19 	& 112.76 	\\ \hline
		G 	& 44.86 	& 11.93 	& 4.09 	& 3.40 	& 1.78 	& 57.61 	\\ \hline
		H 	& 41.66 	& 38.48 	& 4.15 	& 6.10 	& 5.13 	& 38.87 	\\ \hline
		I 	& 44.19 	& 38.93 	& 5.05 	& 6.80 	& 5.41 	& 83.39 	\\
		\hline
		\end{tabular}
	\end{center}
\end{table}

The results in table \ref{tab:results_evolution_results} make one thing immediately clear: Having both a low number of generations and a small population is not good for the search. The one set ($A$) that has low values in both has also scored a far worse fitness than any of the other sets. While the difference is not as high other sets with a low population, but higher number of generations, have also performed worse than their higher-population counterparts. This is shown by sets $C$ and $H$, that both have a population of five.

An interesting thing to note is that $H$ has scored significantly lower than $G$, even though the only difference between them is that their values for generations and population size has been switched around (and parents/children been changed to match). This is likely due to the fact that $H$ is forced to use the same parent (it only chooses one candidate to be the parent for the next generation, just like $A$ and $C$) when creating children for the next generation. All children it generates will be similar to that one parent, which means that any diversity in a generation is ignored in favor of the properties that one parent has. With only one parent, it becomes close to impossible to get out of a local optima (this goes for $A$ and $C$ too).

According to our tests, it appears that a combination of generations and population size such that $\langle generations \rangle \times \langle populationSize \rangle = 250$, assuming $parents >= 2$, appears to be the best when considering how much extra benefit more time gives. $D$, $E$ and $G$ all have this and have scored fairly high considering they have not used more than one minute for their search. The interesting thing is that $G$ has scored the highest fitness of these three, even though it has the fewest number of generations. This suggests that it is more important to have a high population size and a high number of parents per generation than a high number of generations.

The large difference between maximum and average fitness tells the same story as with constrained novelty search: The base map has a large influence in the quality of the map that can be generated.

\begin{table}[!h]
	\begin{center}
	\renewcommand{\arraystretch}{1}
	\caption{Results of the standard genetic algorithm seeded with highest fitness novel individuals.}
	\label{tab:results_novelevolutionhighfitness}
		\begin{tabular}{| K ? S | S | K | S | K | K | K | K |}
		\hline
		Settings Combination & Max Fitness & Average Fitness & Max Fitness (std. dev.) & Conver-gence (average) & Conver-gence (std. dev.) & Evolution Time (seconds) & Novelty Time (seconds) & Total Time (seconds) \\
		\hline
		E + IV   	& 46.34 	& 33.92 	& 2.35 	& 5.80 	& 3.29 	& 41.82 	& 170.88 	& 212.70 	\\ \hline
		F + I 		& 46.99 	& 40.67 	& 2.01 	& 19.10 	& 10.32 	& 101.89 	& 172.18 	& 274.07 	\\ \hline
		G + VIII 	& 45.23 	& 33.94 	& 2.36 	& 3.50 	& 1.65 	& 48.83 	& 344.08 	& 392.91 	\\ \hline
		J + V     	& 45.79 	& 40.96 	& 3.36 	& 6.00 	& 3.86 	& 8.18 	& 335.56 	& 343.74 	\\ \hline
		H + VI   	& 47.70 	& 45.09 	& 3.80 	& 40.10 	& 14.05 	& 39.94 	& 788.76 	& 828.70 	\\ \hline
		I + VII   	& 48.62 	& 44.52 	& 3.20 	& 28.70 	& 15.07 	& 76.44 	& 1291.80 	& 1368.24 	\\ \hline
		D + II    	& 47.09 	& 41.62 	& 3.52 	& 19.30 	& 7.09 	& 39.15 	& 166.25 	& 205.40 	\\ 
		\hline
		\end{tabular}
	\end{center}
\end{table}

\begin{table}[!h]
	\begin{center}
	\renewcommand{\arraystretch}{1}
	\caption{Results of the standard genetic algorithm seeded with highest novelty novel individuals.}
	\label{tab:results_novelevolutionhighnovelty}
		\begin{tabular}{| K ? S | S | K | S | K | K | K | K |}
		\hline
		Settings Combination & Max Fitness & Average Fitness & Max Fitness (std. dev.) & Conver-gence (average) & Conver-gence (std. dev.) & Evolution Time (seconds) & Novelty Time (seconds) & Total Time (seconds) \\
		\hline
		E + IV    	& 46.34 	& 33.92 	& 2.35 	& 5.50 	& 4.09 	& 44.96 	& 162.28 	& 207.24 	\\ \hline
		F + I     	& 47.45 	& 39.46 	& 3.74 	& 5.70 	& 4.30 	& 108.90 	& 160.56 	& 269.46 	\\ \hline
		G + VIII 	& 46.29 	& 34.53 	& 3.25 	& 2.70 	& 1.49 	& 54.08 	& 342.74 	& 396.83 	\\ \hline
		J + V     	& 38.85 	& 34.80 	& 4.03 	& 2.90 	& 2.73 	& 11.64 	& 385.92 	& 397.56 	\\ \hline
		H + VI   	& 43.29 	& 41.24 	& 6.23 	& 6.10 	& 3.67 	& 42.55 	& 878.03 	& 920.58 	\\ \hline
		I + VII   	& 43.55 	& 37.96 	& 4.39 	& 4.80 	& 2.49 	& 78.41 	& 1369.11 	& 1447.53 	\\ \hline
		D + II    	& 42.78 	& 38.14 	& 4.66 	& 7.20 	& 4.71 	& 44.69 	& 182.48 	& 227.17 	\\
		\hline
		\end{tabular}
	\end{center}
\end{table}

The results of seeding the genetic algorithm with maps found by constrained novelty search show the same: There is a clear increase in fitness, no matter if it is seeded with highest novelty or highest fitness individuals. This comes at the cost of having to run the constrained novelty search, however, which at the very least triples the total time needed for a search. The results also show that using the highest fitness individuals is the better choice for most combinations. Seeding with highest fitness individuals also results in a smaller standard deviation among the highest fitness maps generated.

There do no seem to be any real correlation between the time constrained novelty search uses and the final fitness scores, no matter which of the two seedings one look at. Judging from this, it seems that running a "short" constrained novelty search is fine for most seeding purposes.

\subsection*{Novelty of Genetic Algorithm Maps}

Like the constrained novelty search, the maps found by the genetic algorithm are different from each other. This can be seen in figure \ref{fig:results_evolution_samebasediffsettings}, where two novelty maps are shown for the same base map, using different settings ($E$ and $G$). This is a good thing, as it means that the evolution actually finds different solutions instead of being stuck in the same solution.

\insertTwoPicturesLW{0.4}{GA_E_BM5}{GA_G_BM5}{The novelty of two different set of settings ($E$ and $G$ respectively) run on the same base map using the standard genetic algorithm.}{results_evolution_samebasediffsettings}

Looking at the best maps created for each base map (see figure \ref{fig:results_evolution_bestmapsdiffsettings}) they are very different from each other. This is to be expected, as the base maps already are very different (see figure \ref{fig:results_basemapgeneration_novelty} on page \pageref{fig:results_basemapgeneration_novelty}). Though the two searches use  the same base maps, their novelty maps are quite different (with the exception of a few places along the edge of the map).

\insertTwoPicturesLW{0.4}{GA_E_Best}{GA_G_Best}{The novelty of the best maps created from the 10 same base maps with two different set of settings ($E$ and $G$ respectively) using the standard genetic algorithm.}{results_evolution_bestmapsdiffsettings}
\section{Multi-Objective Evolutionary Algorithm}
\label{methodology_moea}
While the standard genetic algorithm seeks to optimize a single objective function, a multi-objective evolutionary algorithms (MOEA) bases its fitness on two or more fitness functions. This is because it is infeasible, for some problems, to combine all interesting features into a single fitness function. An example of this is the basic need for food and water for humans. Most fitness functions would reduce the problem to \textit{amount of food} + {amount of water}. This is infeasible, however, as humans cannot function without water and/or food. Therefore for any solution to be feasible, it is necessary for it to provide both enough food and water for a human to survive. During the evolution process, multi-objective evolutionary algorithms focuses on finding solutions that are \textit{non-dominated solutions} (a solution is dominated when there is at least one other solution "that is better in at least one objective and worse in none."\cite{Togelius2013Controllable}). In the case of food and water for a human, for any solution to dominate another, it must provide at least as much food and water as the solution it dominates, and must provide more of either food or water.

Multiple MOEAs have been suggested over the past twenty years\cite{Deb2001Multi, Fonseca1993Genetic, Srinivas1994Muiltiobjective}. One of the first ones was the \textit{nondominated sorting genetic algorithm (NSGA)}\cite{Srinivas1994Muiltiobjective}, but it was critized for having a \textit{high computational complexity in its sorting}, a \textit{lack of elitism}\footnote{Elitism choses the best candidates from the current generation and carries them over, unchanged, to the next generation.} and the fact that a \textit{sharing parameter} had to be specified. An improved version (\textit{nondominated sorting genetic algorithm II (NSGA-II)}) was proposed by \citeauthor{Deb2000Fast}\cite{Deb2000Fast} in \citeyear{Deb2000Fast}. NSGA-II reduced the overall complexity from $O(M N^3)$ ($M$ being the number of objectives and $N$ being the size of the population) to $O(M N^2)$, added an elitist strategy and made away with the sharing parameter by calculating distances based on the different objectives' parameters.

Using multi-objective evolutionary algorithms is a solid approach to StarCraft map generation, as there are multiple objectives to optimize when generating a good map, such as distance between starting bases, number of choke points, how easily it is to defend a starting position, and the number of expansions available to players. Considering how our primary focus was on the speed-versus-balance-tradeoff, we have chosen to use NSGA-II as the MOEA of our choice.