\chapter{Results}
\label{results}
There are many different options in the search-based map generation approach presented in this thesis. In most cases, these options have an influence on the results generated by the search and/or the runtime of the search. The goal of this thesis is to compare different search methods, so the testing was focused on changing settings that affect the search algorithms instead of the settings that control aspects of the map.

This means that some settings have not been changed during testing. One of them is the fitness calculation settings. They determine how valuable different parts of the fitness function is and changing them between tests would make it really difficult to compare methods. The default values mean that the highest possible fitness, theoretically, is 67. It is very unlikely to be reached, as some parts of the fitness may conflict with each other. The lowest possible fitness score for a map with starting bases is -44.66, and -737 if there are no start bases in a map\footnote{No start bases means that a map literally is unplayable and thus should never be presented as the final result.}.

The cellular automata has a number of settings (e.g. generations, rule sets, map size), which have been kept the same for all tests. The map size is a static 128x128, which is common for smaller two-player maps in StarCraft II. In all tests, the cellular automata ruleset is a basic set of rules that focuses on generating roughly an even amount of each height levels in reasonably sized clusters. The map search options have also been kept the same for all the tests. These options control how many of the different items a map should be able to contain, what the chance of adding or removing an item is and in which area of the map items should be placeable. This means that one half of a map can contain a maximum of one start base, five expansion bases, 18 ramps, eight sets of destructible rocks, and one Xel'Naga tower. 

\section{Base Map Generation}
\label{results_basemapgeneration}

The base maps we create are, as mentioned in section \ref{methodology_ca}, made through the use of a cellular automata. Due to the random seeding we utilize, the quality of the base maps can vary greatly, though most of them are good enough to create StarCraft II maps from.

Shown in figure \ref{fig:results_basemapgeneration_goodbad} are two base maps generated through the cellular automata. The one to the left features four sizable areas of height-level two that all lead down to areas of height-level one. Assuming a start base is placed on either of these areas, the base will be pretty well-defended, as the map requires an attacker to head down to height-level zero to even get to the other side of the map. Any expansion bases are likely to be defendable as well, due to the same properties.

The map on the right is a good example of a bad base map. It has a few small areas of height-level two, all of which connect very poorly to areas of height-level one. This means that start-bases on height-level two will often result in an infeasible map, as it is difficult to create ramps that would result in a path between the two start bases. Start-bases on lower height-levels are less desirable, but not the largest problem this base map has. 

The largest problem is creating a path that links start-bases together almost no matter how they are placed. There are very few places where it is possible to place ramps in such a way that they would connect the top half of the map with the lower half, which greatly reduces the amount of places a start-base can be placed.

\insertTwoPictures{BaseMap_Good}{BaseMap_Bad}{A good (left) and a bad (right) base map.}{results_basemapgeneration_goodbad}

The two base maps shown are very different from each other, a trait which runs true for all the base maps we create. Figure \ref{fig:results_basemapgeneration_novelty} shows how different the generated base maps are. It shows the difference in height-level for every tile for every map. A tile is dark if a specific height-level was encountered on that tile on many maps. The less times the same height-level shows up in the same tile, the lighter the tile is.

\insertPicture{0.8}{BaseMap_Novelty}{A novelty map made based on ten generated base maps.}{results_basemapgeneration_novelty}

From the novelty map, it is easy to see that the base maps generated did not often share the same height-level in the same position.

%do they work?

%show 1 good

% show novelty of base map generation
\section{Constrained Novelty Search}
\label{results_noveltysearch}

The results for the constrained novelty search were gathered by running constrained novelty search with different sets of settings. Table \ref{tab:results_novelty_combinations} describes the different novelty search settings that were used for testing. Some of the settings were used solely for seeding the standard genetic algorithm and the NSGA-II algorithm, and have not been used to test constrained novelty search by itself. In all of the tests, five neighbours are used when calculating the novelty of an individual.

\begin{table}[!h]
	\begin{center}
	\renewcommand{\arraystretch}{1}
	\caption{The constrained novelty search settings.}
	\label{tab:results_novelty_combinations}
		\begin{tabular}{| c ? c | c | c | c | c |}
		\hline
		Settings Label & Generations & Feasible Size & Infeasible Size & Add To Archive \\
		\hline
		I & 30 & 20 & 20 & 3 \\
		\hline
		II & 25 & 25 & 25 & 1 \\
		\hline
		III & 5 & 30 & 30 & 2 \\
		\hline
		IV & 10 & 50 & 50 & 5 \\
		\hline
		V & 20 & 60 & 60 & 1 \\
		\hline
		VI & 50 & 60 & 60 & 1 \\
		\hline
		VII & 50 & 90 & 90 & 1 \\
		\hline
		VIII & 10 & 100 & 100 & 10 \\
		\hline
		\end{tabular}
	\end{center}
\end{table}

Table \ref{tab:results_novelty_results} shows the results of the five sets of settings that were tested with constrained novelty search. Each set of settings was tested on ten different base maps in order to limit how influential a good or bad base map would be for the results. Every one of the ten base maps generated its own set of results. The \textit{average} numbers shown in the table are the average numbers over the ten base maps for that column. \textit{Highest} and \textit{lowest} show the highest and lowest number of acceptable maps for any of the base maps searched with that specific set of settings.

The maps generated by constrained novelty search were evaluated based on whether they were acceptable as a map and how high quality they were. A map is \textit{acceptable} if there is a ground path from the start base to its natural expansion and of \textit{high quality} if its fitness was at least 40.

\begin{table}[!h]
	\begin{center}
	\renewcommand{\arraystretch}{1}
	\caption{Results of constrained novelty search.}
	\label{tab:results_novelty_results}
		\makebox[\textwidth][c]{\begin{tabular}{| c ? K | K | K | K | K | K | K |}
		\hline
		Settings & Accept-able \% (highest) & Accept-able \% (average) & Accept-able \% (lowest) & High-quality \% (highest) & High-quality \% (average) & High-quality \% (lowest) & Time (seconds)\\
		\hline
		I 	& 89.09\% 	& 59.08\% 	& 40.00\% 	& 7.27\% 		& 1.74\% 	& 0.00\% 	& 193.08 	\\ \hline
		II 	& 80.56\% 	& 63.56\% 	& 44.00\% 	& 9.09\% 		& 3.31\% 	& 0.00\% 	& 203.71 	\\ \hline
		III 	& 86.36\% 	& 73.31\% 	& 62.50\% 	& 12.50\% 	& 2.50\% 	& 0.00\% 	& 70.24 	\\ \hline
		IV 	& 71.00\% 	& 62.27\% 	& 46.00\% 	& 8.00\% 		& 2.90\% 	& 0.00\% 	& 207.57  	\\ \hline
		V 	& 84.62\% 	& 65.93\% 	& 54.29\% 	& 8.70\% 		& 2.21\% 	& 0.00\% 	& 210.40 	\\ 
		\hline
		\end{tabular}}
	\end{center}
\end{table}

Set III is the set that generated the most acceptable maps and is also the one with the fewest generations run. This comes from the lowest number of acceptable maps being 8\% to 22.5\% higher than the rest of the sets, not from the highest number of acceptable maps. It does not generate more high-quality maps on average than the other sets, however. This suggests that creating acceptable maps does not necessarily result in creating high-quality maps. 

Another thing to note is the search time between the different sets. The data shows that spending more time on searching does not have much (if any) impact on the final results. This indicates that base maps have a larger influence on the results of the constrained novelty search than time spent searching (after a certain point).

The results also show that the base map a search is performed on has a high influence on the final result. Using set I as an example, a search on one of the ten base maps resulted in 89.09\% acceptable maps, where one of the other base maps resulted in only 40\% acceptable maps. That is less than half of the generated maps that are acceptable. If the base maps did not have much of an influence (i.e. the quality of the base maps was roughly equal), the lowest and highest number of acceptable maps would both be much close to the average value.

High-quality tells the same story. The highest number of high-quality maps found was 7\% or above, but for every single set of settings, there was a base map where no high-quality map was found. This means that constrained novelty search does not guarantee that a high-quality map will be found and that the base map has a lot to say when it comes to finding high-quality maps.

\subsubsection{Novelty of Constrained Novelty Search Maps}

As constrained novelty search attempts to find novel maps, it is important that the search actually produces novel maps. Figure \ref{fig:results_novelty_bm2diffsettings} shows that using different settings for the search on the same base map results in different maps being generated. Figure \ref{fig:results_novelty_bm5diffsettings} further shows that using the same two sets of settings on another base map produces vastly different results. 

It is interesting to note how the concentration of areas change when either the base map or setting is changed. When just switching base map, the maps that have been generated are vastly different. The same happens when another set of settings is used on the same base map.

\insertTwoPicturesLW{0.4}{Novelty_SetI_BM2}{Novelty_SetIII_BM2}{The novelty of two different set of settings ($I$ and $III$ respectively) run on one base map using constrained novelty search.}{results_novelty_bm2diffsettings}

\insertTwoPicturesLW{0.4}{Novelty_SetI_BM5}{Novelty_SetIII_BM5}{The novelty of two different set of settings ($I$ and $III$ respectively) run on another base map using constrained novelty search.}{results_novelty_bm5diffsettings}
\section{Evolution}
\label{Evolution}

Evolutionary programming\cite{IoEC, eiben2002evolutionary} is based on the "survival of the fittest" way of nature. If we can determine the quality of an object through a function (also called the \textit{fitness function}), it is possible to improve on an object over time.

We do this by creating an initial population of objects. From this population, we choose candidates according to the \textit{selection strategy} (usually based on the fitness function) and use them to seed the next generation. When the next generation comes around, \textit{offspring} will be spawned from the chosen candidates (called "parents") and added to the population. Candidates for the next generation are selected from the current population, new offsprings are created and a new generation comes around. This way good candidates will survive and reproduce continuously, until a set quality, or some other condition (e.g. number of generations or barely any quality improvement over the last few generations), is reached.

A simple \textit{selection strategy} would be to always select the highest scoring candidates. The problem with this strategy, is that the evolution may get stuck in a \textit{local optima}. This essentially means that no matter how the best candidates are changed we will never find better solutions and because we always select the best candidates, we never get to try other opportunities. \citeauthor{rocha1999preventing}\cite{rocha1999preventing} suggests that parents for the next population should be chosen using a stochastic method, where a candidate's chance of being selected is based on its ranking compared to other candidates. While it does not completely remove the problem of hitting a local optima, it does alleviate it to some degree.

When it comes to spawning \textit{offspring}, there are two ways to do it\cite[Chapter 2]{PCGBook}. \textit{Recombination} selects two or more parents at random and combine their values (according to some function) to form one or more new offspring. \textit{Mutation} selects a parent and creates an offspring where one or more pieces have been randomly changed (usually within set boundaries). 

While our main idea was to compare MOEA and CNS, we decided that we wanted to see how a normal evolution compared to the other two methods. For this purpose, we implemented an evolutionary algorithm that allows for choosing between two selection strategies (highest fitness and chance based) and the two offspring strategies (mutation and recombination). 

The fitness function we use is discussed in section \ref{MapFitness}. The fitness the EA attempts to optimize is the sum of all the part of the fitness function.
\section{Multi-objective Evolution}
\label{results_moea}

\begin{table}
	\begin{center}
	\renewcommand{\arraystretch}{1}
	\caption{Results of evolution with the NSGA-II algorithm.}
	\label{tab:results_moea_results}
		\begin{tabular}{| c ? c | c | c | p{1.9cm} | p{1.9cm} | p{1cm} |}
		\hline
		Settings & Max fitness & Average fitness & Min fitness & Convergence (average) & Convergence (mode) &  Time \\
		\hline
		A & 31.22 & 0.15 & -94.11 & 2.80 & 1 & 8.43 \\
		\hline
		B & 40.22 & 8.71 & -167.17 & 12.70 & 13 & 57.31 \\
		\hline
		C & 41.30 & 15.19 & -12.30 & 30.50 & 33 & 45.66 \\
		\hline
		D & 43.44 & 12.40 & -91.87 & 20.10 & 22 & 58.05 \\
		\hline
		E & 44.19 & 9.63 & -242.32 & 7.60 & 8 & 73.17 \\
		\hline
		F & 44.63 & 13.29 & -170.02 & 20.10 & 24 & 160.15 \\
		\hline
		G & 43.80 & 2.62 & -384.18 & 3.20 & 4 & 95.97 \\
		\hline
		H & 41.17 & 2.18 & -158.80 & 34.60 & \#N/A & 54.58 \\
		\hline
		I & 44.17 & 7.32 & -98.53 & 28.30 & \#N/A & 115.16 \\


		\hline
		 & & & & & & \\
		\hline
		\end{tabular}
	\end{center}
\end{table}

\begin{table}
	\begin{center}
	\renewcommand{\arraystretch}{1}
	\caption{Results of NSGA-II seeded with highest fitness novel individuals.}
	\label{tab:results_novelevolutionhighfitness}
		\begin{tabular}{| K ? S | S | S | L | L | K | K |}
		\hline
		Settings Combination & Max fitness & Average fitness & Min fitness & Convergence (average) & Convergence (mode) & Evolution Time & Novelty Time \\
		\hline
		E + IV & 46.72 & 7.62 & -454.71 & 5.10 & 6 & 72.54 & 177.82 \\
		\hline
		F + I & 45.71 & 10.96 & -169.30 & 24.70 & 30 & 148.29 & 177.18 \\
		\hline
		G + VIII & 45.81 & 17.10 & -311.90 & 3.40 & 5 & 98.68 & 215.47 \\
		\hline
		J + V & 38.86 & 18.55 & -10.22 & 6.20 & 10 & 12.98 & 341.62 \\
		\hline
		H + VI & 44.23 & 14.40 & -95.39 & 34.70 & 42 & 53.18 & 854.33 \\
		\hline
		I + VII & 43.63 & 12.04 & -167.06 & 28.40 & \#N/A & 113.49 & 1,370.54 \\
		\hline
		D + II & 41.40 & 16.52 & -93.82 & 18.40 & 20 & 55.04 & 168.29 \\


		\hline
		 &  &  &  &  &  &  &  \\
		\hline
		\end{tabular}
	\end{center}
\end{table}

\begin{table}
	\begin{center}
	\renewcommand{\arraystretch}{1}
	\caption{Results of NSGA-II seeded with highest novelty novel individuals.}
	\label{tab:results_novelmoeahighnovelty}
		\begin{tabular}{| K ? S | S | S | L | L | K | K |}
		\hline
		Settings Combination & Max fitness & Average fitness & Min fitness & Convergence (average) & Convergence (mode) & Evolution Time & Novelty Time \\
		\hline
		E + IV & 46.72 & 7.62 & -454.71 & 5.20 & 6 & 72.54 & 177.82 \\
		\hline
		F + I & 46.99 & 14.52 & -310.60 & 18.80 & 25 & 152.42 & 178.18 \\
		\hline
		G + VIII & 46.70 & 10.01 & -525.00 & 2.00 & 1 & 93.81 & 208.26 \\
		\hline
		J + V & 43.69 & 18.96 & -25.01 & 4.90 & 1 & 14.18 & 334.80 \\
		\hline
		H + VI & 45.64 & -36.49 & -305.77 & 36.90 & 24 & 56.51 & 828.85 \\
		\hline
		I + VII & 47.27 & 7.46 & -237.53 & 20.60 & 1 & 118.83 & 1,380.37 \\
		\hline
		D + II & 47.08 & 16.68 & -96.96 & 19.90 & 25 & 62.77 & 174.23 \\


		\hline
		\end{tabular}
	\end{center}
\end{table}