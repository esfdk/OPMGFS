\section{Map Fitness}
\label{methodology_mapfitness}

For the algorithms used in this thesis, some way to determine whether a map was good or not is necessary. The novelty search needs some way of evaluating the maps it finds, in order to actually chose a map at the end of a search. The evolutionary algorithm and multi-objective evolutionary algorithm both need some way (a fitness function) to work properly in the first place.

The evolutionary algorithms were considered the most when figuring out what was important for the fitness function. During evolution a lot of maps are generated and evaluated, so the fitness function had to be fast. Spending even one whole second on evaluating a map would drastically reduce the number of iterations the evolution could run without being too slow (see section \ref{goals_tradeoffs}). While speed was important, a poor fitness function would be at least as bad as a slow one, as it would impact the two evolutions negatively. Therefore, the fitness function had to be both fast and be able to describe the important parts of of a StarCraft II map.

There are three main ways of creating a fitness function\cite{Togelius2010Multiobjective}: \textit{interactive, simulation-based} and \textit{direct}. An interactive fitness function requires humans to give feedback, which is then used to determine whether a map is good or not. A simulation-based fitness function runs one or more simulations of the game on the map and the result of these simulations are used to judge the map. A direct fitness function purely looks at the phenotype and runs calculations on various parameters in order to determine the fitness.

An interactive fitness function was infeasible for the thesis. It would not only require humans that were willing to look at maps and judge them, it would also go against the intention of the project. A simulation-based fitness function would need an AI that could play StarCraft II at an acceptable level. An AI would have to be written for this purpose, but would likely not be very good. Even if a good AI was created, a simulation of a map would be slow due to the nature of the game. A direct fitness function has therefore been used.

In the fitness function, two techniques are used to calculate fitness: Counting of features and the distance between points. For the distance, Jump Point Search\cite{harabor11a, harabor12, Podhraski2013jps} (JPS) is used in order to find a ground path. The length of the path is the number of tiles that make up the path. When the distance between two points is judged, direct distance is also taken into consideration due to the flying units each race has available.

\subsection{Preserving Speed}
\label{methodology_mapfitness_speed}

As mentioned above, the fitness function had to be fast. Some optimizations have been made to the fitness calculations in order to help with this.

When the fitness function is started, it iterates over all cells in the map and saves the position of every start base, other bases and Xel'Naga tower. This process has a time complexity of $O(n\cdot m)$, where $n$ and $m$ is the width and height of the map respectively. Finding these positions at the start saves time later when the positions are needed for calculating different parts of the fitness (see section \ref{methodology_mapfitness_parts}). The path between the two start bases is found, and saved, immediately after as it is used for multiple purposes. This takes roughly 60 milliseconds\footnote{This test was performed on our relatively powerful modern laptops, so results may vary.} due to the use of JPS.

As the map is mirrored, any fitness calculation that involves a start base does not need to be run for both start bases. They will both score the exact same value so such calculations are only run once. This is mainly important when open space around a start base is calculated, as that part relies on using Breadth-First Search, an inherently slow algorithm.

\subsection{Parts of the Fitness Function}
\label{methodology_mapfitness_parts}

The fitness function consists of eleven separate parts. These eleven parts cover the parts of a StarCraft map that generally are considered important. Every part is calculated individually, normalized between an optimal and worst-case value and then multiplied by a significance value based on how important the part is.

	\subsubsection*{Open space around a start base} 
	Open space around the start base is needed to place buildings and be able to maneouver units. The fitness is calculated by selecting the middle point of a start base and using Breadth-First Search to find all traversable tiles no further than 10 tiles from the start base.

	\subsubsection*{The height level the base is located at}
	The higher the start base is positioned, the easier it is to defend. Looking at the maps that are part of the high-level competitive map pool, it is clear that the start bases in those maps are always at the highest level in the map (some maps only have 2 height levels). The fitness is calculated by comparing the hight the start bases are located at to the highest level in the map. 

	\subsubsection*{The path between the two start bases}
	The distance from one start base to another heavily influences which strategies that can be utilized. It should be short enough that rushing is possible, but long enough that rushing is not the only possible winning strategy. The fitness value is the distance of the path between the two start bases. If there is no path, a penalty is given instead (see section \ref{methodology_mapfitness_penalty}).

	\subsubsection*{How many times a new height is reached on the path between the start bases}
	Defending against an attack from high ground offers a sizable advantage. It is therefore important that there is some variation in height when traveling between the two start bases. The fitness is calculated by counting how many times a new height levels is reached on the path between the two bases.

	\subsubsection*{The number of choke points on the path}
	Choke points allow a defender to funnel an attacker's units and prevent the attacker from swarming him. They are important in defenses (and sometimes during attacks), as they can effectively reduce the opponents strength for a while. The fitness value is the number of choke points\footnote{A choke point is any point that has a width of three tiles or less.} that are found on the path between the two start bases.

	\subsubsection*{The distance to the natural expansion}
	A natural expansions is the closest base to a start base. The distance to it should not be long, as players want to be able to get to it quickly. In addition, it is important that players are able to defend both the natural expansion and their starting base without having to divide their army in two. The fitness value is the distance of the path from the start base to the nearest other base. If there is no path, a penalty is given instead (see section \ref{methodology_mapfitness_penalty}).

	\subsubsection*{The distance to the non-natural expansions}
	The non-natural expansions should be spread well around the map. If they are grouped too much, it allows a player to defend all of them at once. Expanding to a new base should, apart from with the natural expansion, mean that it is not that safe. The fitness is calculated by finding the distance of the shortest path to each expansion and normalizing the distance depending on what number of expansion it is.

	\subsubsection*{The number of expansions available}
	Expansions are necessary for players to continue building their economy. There should be enough expansions that players will not run out of resources too fast, but also enough that they will not have to fight over every expansions because there are too few in the map. The fitness value is the number of expansions divided by the number of start bases.

	\subsubsection*{The placement of Xel'Naga towers}
	Xel'Naga towers provide unobstructed vision of an area in a radius of 22 tiles centered on the tower when a player has a unit net to the tower. They should offer a view of the main path between the start bases in order to be a valuable asset early on. The fitness is the number of tiles of the path between the start bases a Xel'Naga tower covers.

	\subsubsection*{How open the start bases and expansions are}
	How open a base is determines how many directions it can be approached from. A base should not be completely open, as it would be too difficult to defend properly. The fitness here consists of two parts: How many tiles in an area around the start base that are not blocked, and how many of the compass' directions (north, north-east, etc.) that are open in a direct line.

\subsection{Penalty Function}
\label{methodology_mapfitness_penalty}
There are some maps that are infeasible for playing, e.g. maps with no ground path between the start bases. Such maps should not dominate the fitness rankings, as it would lead to a lot of infeasible maps being generated. 

Infeasible maps receive a penalty that reduces their fitness. A \textit{death penalty}\cite{coello2012constraint} is not used, however, as an infeasible map may have some interesting features if one looks away from what makes them infeasible. Instead, the penalty for the map is calculated as follows.

\begin{equation}
	penalty := \frac{\langle maxTotalFitness\rangle}{3}
\end{equation}

This penalty ensures that the map's fitness will not dominate feasible maps, but it may still be chosen over feasible maps that are really bad.