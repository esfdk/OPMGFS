\section{Constrained Novelty Search}
\label{methodology_csn}
While objective-based evolutionary algorithms seek to optimize one or more objectives in order to reach an optimal solution, novelty search seeks to optimize diversity and novelty. Novelty search is a recent algorithm that evolves like any evolutionary algorithm, but instead of basing the survivor selection on objective fitness, novelty search selects survivors based on their novelty, thus diversifying the population\cite{liapis2014constrained}. A novel archive keeps track of novel solutions from each generation, ensuring that solutions are novel both compared to current and historic behaviour. The main benefit of novelty search is that it avoids the deception and local optima problems that is frequently experienced in objective optimization approaches\cite{lehman2011abandoning}.

Constrained novelty search applies constraints to the novelty search algorithm in order to direct evolution towards a desired outcome\cite{liapis2014constrained}. The algorithm uses the concept of \textit{two-population novelty search} introduced by \citet{lehman2010revising}. \textit{Two-population novelty search} maintains one population of \textit{feasible} and another for \textit{infeasible} individuals. 
The two types of individuals are kept separate as it is preferable to avoid comparisons between them\cite{liapis2014constrained} as a constrained novelty search will quickly kill off infeasible solutions if they are compared to feasible solutions. This is not desirable behaviour as it is likely that the infeasible solutions can have feasible children when performing novelty searches. Instead, any feasible child of an infeasible solution is migrated to the feasible population and vice versa. 

\citet{liapis2014constrained} presents two different versions of constrained novelty search: \textit{Feasible-infeasible novelty search} (FINS) and \textit{feasible-infeasible dual novelty search} (FI2NS). Feasible-infeasible novelty search attempts to maximize the novelty score of feasible individuals and minimizing the distance of infeasible individuals from feasibility. In feasible-infeasible dual novelty search, each population performs novelty search separately and contains their own separate feasible and infeasible populations. FINS with the suggested \textit{offspring boost} enhancement was chosen for this project.

\subsection{Distance to Feasibility Measure}
\label{methodology_csn_distance}
The distance to feasibility for the representation used in this thesis is measured in two parts. The first part is to calculate the number of map points that were successfully \textit{placed} during map conversion. If there are too few or too many placed map points of a given type, the distance to feasibility is increased by a penalty value multiplied by the number of items missing and in excess. The second part is to check if there is a path between the start bases; if no ground path is found, the map is not feasible as a StarCraft II map (too many strategies would be excluded), and a penalty is applied.

\subsection{Novelty Measure}
\label{methodology_csn_measure}
The novelty of an individual is the average distance between its k-nearest neighbours in both the novel archive and the current population. To calculate the distance between two solutions \textit{A} and \textit{B} in the context of this thesis, each map point in solution \textit{A} is compared to every map point in solution \textit{B}. The absolute difference in angle and distance (distance from centre of map, not novelty) of each of these comparisons are summed to form the distance between the two solutions.