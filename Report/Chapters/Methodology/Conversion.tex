\section{Converting Representations To Maps}
\label{methodology_conversion}
Converting an individual representation into a StarCraft map requires placing the map points of the representation on the base map that is connected to the representation. The conversion process consists of iterating over the set of map points and attempting to place each map point onto the base map. Depending on a number of factors, placement may not be possible, and so another attempt is made at a slightly displaced location, e.g. one tile directly to the west. If placement is still not successful, or if angle and/or distance values of the map point are outside their constraints, a map point will be registered as \textit{not placed}. Any map point that is registered as having not been placed does not count for fitness or feasibility calculations (but does count for novelty calculation). When the entire set of map points have been iterated over, the completely map is created by mirroring the top half, placing and smoothing cliffs, and finally mirroring the top half again\footnote{This second mirroring happens due to cliff placement often creating a symmetric line through the middle.}. 

A map point may fail to be placed due to either an area being occupied by another map point (prevents overlapping map points, e.g. bases) or another placement constraint preventing placement (e.g. ramps cannot be placed if the section of cliff is not wide enough). If placement is possible, the map point is placed by flattening the area around the map point, so that the entire area is the same height as the centre tile of the map point. The area is then marked as \textit{occupied}, such that no other map point can overlap with the placed map point. The exceptions to this method are the destructible rocks and ramps. 

A \textbf{StartBase} map point flattens and occupies a 24x24 area, places a 5x5 starting base in the centre of that area, and places blue minerals and gas geysers around the starting base in a specific pattern. \textbf{StartBase} map points are always placed first, as they are required for the map to be feasible. A \textbf{Base} flattens and occupies a 16x16 area, places a base marker (used as an indicator on the map) slightly to the south-west of the centre of the area, and places blue minerals and gas geysers in the same pattern as the start base. A \textbf{GoldBase} is the same as a regular base, except the minerals are gold instead of blue. A \textbf{XelNagaTower} flattens and occupies a 4x4 area, in which the tower itself is placed. \textbf{DestructibleRocks} do not occupy or flatten an area, but can be placed on already occupied areas. However, they cannot be placed on cliffs, impassable terrain, start bases, Xel'Naga towers, minerals, or gas. Upon placement, they are randomly chosen to be either 2x2, 4x4, or 6x6 tiles large. \textbf{Ramp} placement is handled slightly different from the other map points. Because it is quite unlikely that a cliff will be at the exact spot calculated from the angle and distance of the map point (and locating the nearest cliff is inefficient), ramp positions are chosen using a hash function of the angle and distance to select a position from a set of known cliff positions in the map.